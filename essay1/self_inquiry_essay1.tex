%!TEX TS-program = lualatex
\documentclass[12pt]{article}
\usepackage[a4paper, total={7in, 10in}]{geometry}
\usepackage{apacite}
\usepackage{natbib}
\usepackage{fontspec}
\setmainfont{Times}
\title{Self Inquiry Essay 1}
\author{Steven Hé (Sīchàng)}
\begin{document}
\maketitle
We discussed the importance of family for individuals and society,
but is family essential?

I am seeing more and more arguments that suggest that monogamy is not natural
for human beings.
In \textit{What is the family? Is it universal?} \citep{gittins1993family},
the authors stated that ``it has been estimated that only 10 persent of all
marriages in the world are actually monogamous'' and
``serial monogamy is becoming increasingly common in our own [society]''
\citep{gittins1993family}.
Serial monogamy is still at the boundaries of what would be commonly considered
ethical in our modern societies.
With that and the estimated low percentage of monogamous marriages,
it seems to me that the majority of people are simply not satisfied with a
single partner and
more and more are trying out serial monogamy as a potential solution.

While serial monogamy is closer to what most people accept,
what they truly want may instead be polyamory.
Polyamory refers to the state or will of having multiple romantic partners,
where each partner has such consent, according to \cite{wikipedia2023polyamory}.
In their podcast \citep{fridman2023aella},
Fridman argues that polyamory is closer to human nature than monogamy is;
as supportive evidence, Aella describes her massive relationship survey,
where the results show that both male and female cheat a lot on their partners.
Such results urge us to think about whether modern culture and what is seen as
the norm is in fact against the nature and is causing negative effects.
\pagebreak
\bibliographystyle{apacite}
\begin{raggedright}
\bibliography{Bibliography}
\end{raggedright}
\end{document}