%!TEX TS-program = lualatex
\documentclass[12pt]{article}
\usepackage[a4paper, total={7in, 10in}]{geometry}
\usepackage{url}
\usepackage{apacite}
\usepackage{natbib}
\usepackage{fontspec}
\setmainfont{Times}
\title{Self Inquiry Essay 1}
\author{Steven Hé (Sīchàng)}
\begin{document}
\maketitle
We discussed the importance of family for individuals and society,
but is family essential?

In class \citep{pouille2023ethics},
Professor Pouille argues that family is fundamental to humanity:
For individuals, family is their anchor point to society,
where they find their place;
for society, family is the basic functional unit.
I agree that, as the one common unit for economic, reproduction, and social
activities, family is arguably what held humanity up in the pass and even at
present.
For example, as Lu points out in class \citep{pouille2023ethics},
many drug dealers' children enter this vicious circle where
they participate in drug dealing themselves because their parents are in prison
and drug dealing becomes their only source of financial income.
This is a negative example where broken families lead individuals to crimes.
As a positive example, \citet{lu1998family} indicate that
younger Taiwanese generations learn the family roles and responsibilities from
older generations via their family education and
continue to enjoy playing the roles in their families as they grow up.
As these examples show, having a normal family is crucial for any individual to
live a life that is commonly accepted by society and help keep society as a
whole on track.

For society to stay on track, ``the four basic functions of the family,''
as summarized by \citet[p. 60]{gittins1993family}—``common residence;
economic co-operation; reproduction; sexuality''—are indeed fundamental.
Starting from the bottom, sexuality is the basis of reproduction.
Besides that, sexuality in traditional families has provided individuals
a means to release their energy and sexual desires,
with negative examples of incidents where family-less men rape women.
Moving up, reproduction is naturally fundamental to the survival of human as
a species, just like common residence is naturally fundamental to the survival
of human as social animals.
On top of all that, economy is how individuals gain food and shelters and
is fundamental to society.
So, the aforementioned four functions of the family are fundamental to society.

However, as we can see from the changes in the real world,
the traditional ideology of family is challenged.

I am seeing more and more arguments that suggest that monogamy is not natural
for human beings.
In \textit{What is the family? Is it universal?}, \citet[p. 63]{gittins1993family}
state that ``it has been estimated that only 10 persent of all
marriages in the world are actually monogamous'' and
``serial monogamy is becoming increasingly common in our own [society]''.
Serial monogamy is still at the boundaries of what would be commonly considered
ethical in modern society.
With that and the estimated low percentage of monogamous marriages,
it seems to me that the majority of people are simply not satisfied with a
single partner and
more and more are trying out serial monogamy as a potential solution.

While serial monogamy is closer to what most people accept,
what they truly want may instead be polyamory.
Polyamory refers to the state or will of having multiple romantic partners,
where each partner has such consent, according to \cite{wikipedia2023polyamory}.
In their podcast \citep{fridman2023aella},
Fridman argues that polyamory is closer to human nature than monogamy is;
as supportive evidence, Aella describes her massive relationship survey,
where the results show that both male and female cheat a lot on their partners.
Such results urge us to think about whether modern culture and what is seen as
the norm is in fact against human nature and is causing negative effects.

\citep{schwitzgebel2008thoughts}

\citep{malinowska2022love}
\pagebreak
\bibliographystyle{apacite}
\bibliography{Bibliography}
\end{document}