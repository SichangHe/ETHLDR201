%!TEX TS-program = lualatex
\documentclass[12pt]{article}
\usepackage[a4paper, total={7in, 10in}]{geometry}
\usepackage{url}
\usepackage{apacite}
\usepackage{natbib}
\usepackage{fontspec}
\setmainfont{Times}
\title{Self Inquiry Essay 1\\—Is family essential?}
\author{Steven Hé (Sīchàng)}
\begin{document}
\maketitle
We discussed the importance of family for individuals and society
in class \citep{pouille2023ethics},
but is family essential for society?

Indeed, family seems to be essential for society.
In class \citep{pouille2023ethics},
Professor Pouille argues that family is fundamental to humanity:
For individuals, family is their anchor point to society,
where they find their place;
for society, family is the basic functional unit.
I agree that family is arguably what held humanity up in the past and even at
present.
For example, as Lu points out in class \citep{pouille2023ethics},
many drug dealers' children enter this vicious circle where
they participate in drug dealing themselves because their parents are in prison
and drug dealing becomes their only source of financial income,
and they grow up to repeat their parents' path.
This is a negative example where broken families lead individuals to crimes.
As a positive example, \citet{lu1998family} indicate that
younger Taiwanese generations learn the family roles and responsibilities from
older generations via their family education and
continue to enjoy playing the roles in their families as they grow up.
As these examples show, having a normal family is crucial for any individual to
live a life that is commonly accepted by society and to help keep society as a
whole on track.

What are the deep reasons why family is so important for society?

Family is essential, probably in that it provides necessary functions
that are essential for society.

For society to stay on track, ``the four basic functions of the family,''
as summarized by \citet[p. 60]{gittins1993family}—``common residence;
economic co-operation; reproduction; sexuality''—are indeed fundamental.
Starting from the bottom, sexuality is the basis of reproduction.
Besides that, sexuality in traditional families has provided individuals
a means to release their energy and satisfy their sexual desires,
with negative examples of incidents where family-less men rape women.
Moving up, reproduction is naturally fundamental to the survival of human as
a species, just like common residence is naturally fundamental to the survival
of human as social animals.
On top of all that, the economy is how individuals gain food and shelter and
is fundamental to society.
So, the aforementioned four functions of the family are fundamental to society.

So, is family essential because it is the only way that the fundamentals of
society are satisfied?
At first glance, it seems to be the only functional structure that provides
``common residence; economic co-operation; reproduction; sexuality''
\citep[p. 60]{gittins1993family}.

However, as we can see from the changes in the real world,
the traditional ideology of family is challenged.
As more and more people are satisfied to live a life while
practicing against the traditional ideology of family,
it seems that the ideology of family itself is in fact inessential.

One of the major challenges to traditional families
in modern society is the questioning of monogamy.

Traditional families in modern society are typically monogamous.
\citet[pp. 4]{schwitzgebel2008thoughts}, for example, claims that
``[i]n conjugal love,
one commits oneself to seeing one's life always with the other in view.''
Couples vow that they will be loyal to their partners and not have other
romantic relationships when they marry.
With that, we can say monogamy is at the core of traditional family.

So, challenging monogamy is challenging the ideology of family.
Therefore, the denial of monogamy would indicate that family is inessential.

How is monogamy challenged?

I am seeing more and more arguments that suggest that monogamy is not natural
for human beings.
\citet[p. 63]{gittins1993family} state that
``it has been estimated that only 10 percent of all
marriages in the world are actually monogamous'' and
``serial monogamy is becoming increasingly common in our own [society]''.
Serial monogamy is still at the boundaries of what would be commonly considered
ethical in modern society.
With that and the estimated low percentage of monogamous marriages,
it seems to me that the majority of people are simply not satisfied with a
single partner and
more and more are trying out serial monogamy as a potential solution.

Why do we have a low monogamy rate and an increasing percentage of serial
monogamy?

While serial monogamy is closer to what most people accept,
what they truly want may instead be polyamory.
Polyamory refers to the state or will of having multiple romantic partners,
where each partner has such consent, according to \cite{wikipedia2023polyamory}.
In their podcast \citep{fridman2023aella},
Fridman argues that polyamory is closer to human nature than monogamy is;
as supportive evidence, Aella describes her massive relationship survey,
where the results show that both males and females cheat a lot on their
partners.
Such results urge us to think about whether modern culture and what is seen as
the norm is in fact against human nature and is causing negative effects.

So, the challenges to monogamy suggest that although the ideology of family
provides a viable solution for the four functions fundamental to society,
it is a solution that is against human nature and
causes negative effects on society.
The examples of polyamory also provide an alternative to family that may
satisfy the four functions.

That was more of a challenge from an ideological approach.
How is the ideology of family challenged in practice?

The challenge to the traditional ideology of family in practice comes from
the rise of modern technology.
\citet{malinowska2022love} argues that the way we love is
dramatically changed by modern technology.
She summarizes that dating platforms and other technologies are redirecting
people from traditional love to a commercial product of sex,
where people shop for appealing and low-risk sex \citep{malinowska2022love}.
Instead of looking for one romantic partner to deeply invest in them and form a
family, some people seek quick, low-cost,
and perhaps casual sexual relationships and take their time to form a family.
This phenomenon has been around for a long time.
I see it reflected in the media such as in movies.
I also see it in real-life stories such as long-time boyfriends and girlfriends.
So, I think it is a valid point that technology is challenging the traditional
ideology of family in that it promotes people to engage in less involved
relationships instead of forming a traditional family.

So, is family inessential to society?
I think family in general is still essential to society.
Despite the low percentage of monogamous marriage mentioned by
\citet[p. 63]{gittins1993family} and the high cheating rate between couples
shown by \citet{fridman2023aella},
most of the workforce are still trying to maintain a family that is acceptable
by society and helping themselves and society stay on track.
Family is still the most prominent way where the four functions mentioned by
\citet[p. 60]{gittins1993family} are achieved.
But, as shown in the challenges to monogamy and love,
there are some alternatives to family,
so the ideology of family is most likely not essential to society.
Until any alternative ideology goes mainstream, though,
family will probably still be the one structure that holds society up—family
is by far essential.

(1136 words).
\pagebreak
\bibliographystyle{apacite}
\bibliography{Bibliography}
\end{document}